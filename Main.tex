\documentclass[
	ngerman,
    %draft, %Dient zur Erkennung von 'overfull-Boxen' + Verschnellert das Kompilierens, da Bilder nicht aufgelöst werden + Keine Links
	toc=listof, % Abbildungsverzeichnis sowie Tabellenverzeichnis in das Inhaltsverzeichnis aufnehmen
	toc=bibliography, % Literaturverzeichnis in das Inhaltsverzeichnis aufnehmen
	footnotes=multiple, % Trennen von direkt aufeinander folgenden Fußnoten
	parskip=half, % vertikalen Abstand zwischen Absätzen verwenden anstatt horizontale Einrückung von Folgeabsätzen
	numbers=noendperiod, % Den letzten Punkt nach einer Nummerierung entfernen (nach DIN 5008)
]{scrartcl}


%Metadaten der Dokumentation

%% Projektdaten ------------------------------------------------------
%Projekttitel 
\newcommand{\projektKurztitel}{Kurztitel\xspace}
\newcommand{\projektLangtitel}{Langtitel\xspace}


%Hausarbeit etc.
\newcommand{\kategorie}{KategorieDerArbeit\xspace}

\newcommand{\startTerminProjekt}{24.06.2025}
\newcommand{\abgabetermin}{01.01.9999}
\newcommand{\pruefungsID}{XXXXX\_P}


%%Autoren ------------------------------------------------------
%Autor One definieren
\newcommand{\autorVornameOne}{Max\xspace}
\newcommand{\autorNachnameOne}{Mustermann\xspace}
\newcommand{\autorFullNameOne}{\autorVornameOne \autorNachnameOne\xspace}
\newcommand{\autorFullNameOneOhneLeerzeichen}{\autorVornameOne \autorNachnameOne}
\newcommand{\autorAnschriftOne}{Musterstraße 13\xspace}
\newcommand{\autorOrtOne}{12345 Musterstadt\xspace}


%Weitere Daten Autor One
\newcommand{\autorMatrikelnummerOne}{12345\xspace}
\newcommand{\autorSemesterNummerOne}{-1\xspace}
\newcommand{\autorHandyPrivatOne}{123456789012\xspace}
\newcommand{\autorMailPrivatOne}{max-mustermann@domain.de\xspace}
\newcommand{\autorMailWorkOne}{muster@mail.de\xspace}

%Weiteres
\newcommand{\autorenFullNameRow}{\autorFullNameOne}



%Ansprechpartner / Dozenten ------------------------------------------------------
\newcommand{\ansprechpartnerOne}{Prof. Dr.-Ing. Max Mustermann\xspace}
\newcommand{\ansprechpartnerTwo}{Erika Mustermann \xspace}

\newcommand{\ansprechpartnerRow}{\ansprechpartnerOne, \ansprechpartnerTwo}



%%Hochschule ------------------------------------------------------
\newcommand{\studiengang}{NameStudiengang\xspace}
\newcommand{\vorlesungTitel}{VorlesungTitel\xspace}


%Hochschule Logos
\newcommand{\betriebHSHarzShort}{02_Datein/01_Bilder/00_Base/LogoHSHarz_Short.png}
\newcommand{\betriebHSHarzFull}{02_Datein/01_Bilder/00_Base/LogoHochschule_Full.png}


\newcommand{\eidesstattlicheErklaerungDatum}{\today}
\newcommand{\eingereichtAmDatum}{\today} % Metadaten zu diesem Dokument (Autor usw.)
%Für Sonderzeichen und Umlaute
\usepackage[T1]{fontenc}
\usepackage[utf8]{inputenc}

%Schrift ändern
\usepackage[scaled]{uarial} %Klon von Arial - mit Serifen

\renewcommand\familydefault{\sfdefault} %Default-Schrift auf "ohne Serifen" umstellen

% Schriftgröße zulassen
\usepackage{anyfontsize}

% Dokument auf Deutsch umstellen
\usepackage{babel}

% Tabellen
\PassOptionsToPackage{table}{xcolor}
\usepackage{tabularx}
\usepackage{colortbl}
\usepackage{float} %Damit kann man den Parameter "H" bei der table Umgebung verwenden und die Tabelle bleibt an der Stelle.

% Tabellen Linien formatieren
\usepackage{tabularx}
\usepackage{bigstrut}

% Einfache Definition der Zeilenabstände und Seitenränder etc.
\usepackage{setspace}
\usepackage{geometry}
%\usepackage[showframe]{geometry} %Zeigt einen Rahmen um den Text. Für Überlappungen.

% Einstellungen für PDF-Dateien
\usepackage{pdfpages}
\usepackage[
    bookmarks,
    bookmarksnumbered,
    bookmarksopen=true,
    bookmarksopenlevel=1,
    colorlinks=true,
    % di%ese Farbdefinitionen zeichnen Links im PDF farblich aus
    anchorcolor=MainColor,% Ankertext
    citecolor=MainColor, % Verweise auf Literaturverzeichniseinträge im Text
    filecolor=MainColor, % Verknüpfungen, die lokale Dateien öffnen
    menucolor=MainColor, % Acrobat-Menüpunkte
    urlcolor=MainColor,
    linkcolor=MainColor,
    %%
    pdftex,
    plainpages=false, % zur korrekten Erstellung der Bookmarks
    pdfpagelabels=true, % zur korrekten Erstellung der Bookmarks
    hypertexnames=false, % zur korrekten Erstellung der Bookmarks
    linktoc=page,
]{hyperref}
\hypersetup{
    pdftitle={\projektKurztitel -- \projektLangtitel},
    pdfauthor={\autorFullNameOne},
    pdfcreator={\autorFullNameOne},
    pdfsubject={\projektKurztitel -- \projektLangtitel},
    pdfkeywords={\projektKurztitel -- \projektLangtitel},
}

% Einbinden von Bilder (.jpg || .png) ermöglichen
\usepackage{graphicx}

\usepackage[automake, acronym, toc, nonumberlist, nopostdot, nogroupskip, style=super]{glossaries}
\makeglossaries

%Fußzeile und Kopfzeile 
\usepackage{chngcntr} % fortlaufendes Durchnummerieren der Fußnoten

%Literaturverzeichnis
\usepackage[babel,german=quotes]{csquotes} %Empfohlen
\usepackage[backend=biber,
    style=authoryear,isbn=false
]{biblatex}

%Löscht den Inhalt des Feldes 'note'
\AtEveryBibitem{%
    \clearfield{note}%
}

%URL-Umbrüche anpassen
\setcounter{biburllcpenalty}{7000}
\setcounter{biburlucpenalty}{8000}

\addbibresource{04_Literaturverzeichnis/Literaturverzeichnis.bib}

% Für die Darstellung eines Euro-Zeichens.
\usepackage{eurosym}

% Für das Titelblatt 
\usepackage[pages=some]{background}

%Einbinden von einer PDF-Datei
\usepackage{pdfpages}

%Sonstiges
\usepackage{xspace} % sorgt dafür, dass Leerzeichen hinter parameterlosen Makros nicht als Makroendezeichen interpretiert werden

\usepackage[normalem]{ulem} % unterstreichen [\uline], durchstreichen [\sout] etc. % verwendete Packages
\input{03_Settings/02_Projektspezifisch/01_ProjektspezifischePackages.tex}
\newcommand{\addAbbildungQuelle}[5] % Parameter: scale, caption, label, file, Quelle
{

    \begin{minipage}{0.9\textwidth}
        \centering

        \includegraphics[scale=#1]{02_Datein/02_Abbildungen/#4}

        \captionof{figure}[#2]{#2; Quelle: #5}
        \label{#3}

    \end{minipage}

}

\newcommand{\addTabelleScaleTextQuelle}[5] % Parameter: scale (z.B. \columnwidth), caption, label, file, Quelle
{
    \begin{table}[H]
    \centering
    \captionof{table}[#2]{#2; Quelle: #5}
    \label{#3}
    \begin{#1}
    \input{02_Datein/03_Tabellen/#4}
    \end{#1}

    \end{table}
}

% Weiteres

\newcommand{\Vgl}{Vgl.\xspace}

\newcommand{\Anfuehrungsstriche}[1] %inhalt
{\glqq #1\grqq\xspace}

\newcommand{\ZumBeispiel}{z.B.\xspace}

\newcommand{\Gendern}[2]{#1:#2\xspace} %zur Vereinfachung von Prozessen
%Seitenränder
\setlength{\topskip}{\ht\strutbox} % behebt Warnung von geometry
\geometry{a4paper,left=25mm,right=20mm,top=25mm,bottom=35mm}

\usepackage[
	automark, % Kapitelangaben in Kopfzeile automatisch erstellen
	headsepline, % Trennlinie unter Kopfzeile
	ilines % Trennlinie linksbündig ausrichten
]{scrlayer-scrpage}


% Kopfzeile 	---------------------------------------------------------------------
\pagestyle{scrheadings}
\clearpairofpagestyles

%Linker Teil der Kopfzeile
\renewcommand{\headfont}{\normalfont} % Schrift der Kopfzeile
\ihead{\large{\projektKurztitel}\\
	\begin{minipage}{0.93\textwidth}
		\small{\projektLangtitel} \\
	\end{minipage} \\
	\headmark}

%Rechter Teil der Kopfzeile
\chead{}
\ohead{\includegraphics[scale=0.12]{\betriebHSHarzShort}}
\setlength{\headheight}{16.2mm} % Höhe der Kopfzeile

% Fußzeile----------------------------------------------------------------------
\ifoot{\autorFullNameOne}
\cfoot{}
\ofoot{\pagemark}

% Schusterjungen und Hurenkinder vermeiden
\clubpenalty = 10000
\widowpenalty = 10000
\displaywidowpenalty = 10000

% Vermeidet das Fußnoten aufgeteilt werden!
\interfootnotelinepenalty=10000


% Abstand zwischen Nummerierung und Überschrift definieren
\newcommand{\colTableLightBlueSpace}{1.5cm}

% Abschnittsüberschriften im selben Stil wie beim Inhaltsverzeichnis einrücken
\renewcommand*{\othersectionlevelsformat}[3]{
	\makebox[\colTableLightBlueSpace][l]{#3\autodot}
}

%Einrückung des Inhaltsverzeichnisses
\RedeclareSectionCommand[tocindent=0em,tocnumwidth=1em]{section}
\RedeclareSectionCommand[tocindent=1em,tocnumwidth=2em]{subsection}
\RedeclareSectionCommand[tocindent=3em,tocnumwidth=3em]{subsubsection}


%Aufzählungszeichen definieren
\renewcommand{\labelitemi}{\small$\blacktriangleright$}
\renewcommand{\labelitemii}{\small$\blacktriangleright$} % Definitionen zum Aussehen der Seiten
%Farben für Links
\usepackage{xcolor}

\definecolor{MainColor}{RGB}{0,107,179}

\definecolor{colTableLightBlue}{RGB}{0,107,179} 


% Möglichkeit der Erweiterung

%Dummy-Text (Befehl \lipsum[<par-range>]; \lipsum[<par-range>][<sentence-range>])
\usepackage{lipsum}
% Möglichkeit der Erweiterung

\definecolor{colorBeispiel}{RGB}{255,255,255} %Weiß



\begin{document}

%Deckblatt einbinden ------------------------------------------------------
\phantomsection
\pdfbookmark[1]{Deckblatt}{deckblatt}
\begin{titlepage}


    %%Titelseite konfigurieren
    \backgroundsetup{
        scale=1,
        color=black,
        opacity=1,
        angle=0,
        contents={%
                \includegraphics[width=\paperwidth,height=\paperheight]{02_Datein/01_Bilder/00_Base/background_hsHarz.png}
            }%
    }

    %Titelseite anzeigen
    \BgThispage

    \vspace*{16ex} %Mit Exposé-Überschrift 16, sonst 20
    \begin{center}
        \huge{\textbf{\kategorie}}\\
        \vspace*{6ex}

        \huge{\textbf{\projektKurztitel}}\\
        \huge{\projektLangtitel}\\
        \vspace*{6ex}

        \LARGE
        {
            angefertigt an der Hochschule Harz \\ Fachbereich Verwaltungswissenschaften
            \\~\\
            Studiengang: \studiengang
        }

    \end{center}

    \vfill

    %Abstand der Tabelleneinträge 
    \renewcommand{\arraystretch}{1.0}
    \begin{table}[h]
        %Schriftgröße anpassen
        \begin{large}
            %Spaltenbreite anpassen, damit das centering passt
            \begin{tabular}{p{0.5\textwidth}p{0.44\textwidth}}
                \textbf{vorgelegt von:}               & \textbf{angefertigt bei:} \\
                                                      &                           \\
                \autorFullNameOne                     & Erstbetreuer              \\
                \autorAnschriftOne                    & \ansprechpartnerOne       \\
                \autorOrtOne                          &                           \\
                %\autorHandyPrivatOne & \\
                %\autorMailPrivatOne & \\
                                                      &                           \\
                Semester: \autorSemesterNummerOne     & Zweitbetreuer             \\
                Matrikel-Nr.: \autorMatrikelnummerOne & \ansprechpartnerTwo       \\
                E-Mail: \autorMailWorkOne             &                           \\
                                                      &                           \\
                                                      &                           \\
                \textbf{Eingereicht am:} \eingereichtAmDatum
            \end{tabular}
        \end{large}
    \end{table}



\end{titlepage}
\cleardoublepage

%Inhaltsverzeichnis --------------------------------------------------------
\phantomsection
\pagenumbering{Roman}
\pdfbookmark[1]{Inhaltsverzeichnis}{inhalt}
\tableofcontents
\cleardoublepage

%Abkürzungsverzeichnis ----------------------------------------------------- 
% Es werden nur die Abkürzungen aufgelistet, die im Text verwendet werden

\newacronym{itsys}{IT-SYS}{IT-Systemhaus der Bundesagentur für Arbeit}
\setlength{\glsdescwidth}{0.7\textwidth} % Spalte 2 in der Breite anpassen
\renewcommand{\glsnamefont}[1]{\textbf{\textcolor{MainColor}{#1}}} %Abkürzungen fett und farbig.
\renewcommand*{\arraystretch}{1.4}% Abstand zwischen den Einträgen
\printglossary[type=\acronymtype,title={Abkürzungsverzeichnis}]
\clearpage

%Abbildungsverzeichnis -----------------------------------------------------
\phantomsection
\listoffigures
\cleardoublepage

%Tabellenverzeichnis -------------------------------------------------------
\phantomsection
\listoftables
\cleardoublepage


%Inhalt ---------------------------------------------------------------------
\pagenumbering{arabic}
\onehalfspacing %Anpassung an Rahmenbedingungen
\input{01_Inhalt/01_Einleitung/01_Einleitung.tex}
\subsection{Kapitel in Einleitung}
\label{subsec:Kapitel_in_Einleitung}

Hier stehen Informationen über das \acrshort{itsys}.\footnote{\Vgl \cite[1]{01Kategorie_QUELLE_DasITSystemhausderBundesagenturfurArbeit}.}

Hier ist eine Abbildungen:

\addAbbildungQuelle{0.5}{Vereinfachtes Organigramm der Bundesagentur für Arbeit (Zentrale)}{abb:Vereinfachte_Organigramm_BA}{Vereinfachte_Organigramm_BA.png}{In Anlehnung an \cite{02Kategorie_QUELLE_OrganigrammderZentrale}}
\subsubsection{Unterkapitel in Einleitung}
\label{subsec:Unterkapitel_in_Einleitung}

\addTabelleScaleTextQuelle{normalsize}{Vereinfachte Zeitplanung}{tab:VereinfachteZeitplanung}{VereinfachteZeitplanung.tex}{Eigene Darstellung}

\autoref{tab:VereinfachteZeitplanung} zeigt die vereinfachte Zeitplanung.


\clearpage

%Literatur ------------------------------------------------------------------

\clearpage
\setlength{\bibitemsep}{6pt}
\renewcommand{\refname}{Literaturverzeichnis} %Überschrift abändern
\printbibliography

%Eidesstattliche Erklärung --------------------------------------------------
\clearpage

\addsec{Eidesstattliche Erklärung}

Ich, \autorFullNameOneOhneLeerzeichen, versichere hiermit, dass ich meine \textbf{\kategorie} zum Thema
\begin{quote}
    \begin{center}
        \textbf{\projektKurztitel} \\
        \textit{\projektLangtitel}
    \end{center}
\end{quote}
selbständig verfasst und keine anderen als die angegebenen Quellen und Hilfsmittel benutzt habe,
wobei ich alle wörtlichen und sinngemäßen Zitate als solche gekennzeichnet habe. Die Arbeit
wurde bisher keiner anderen Prüfungsbehörde vorgelegt und auch nicht veröffentlicht.

\vspace{0.8cm}

\autorOrtOne, den \eidesstattlicheErklaerungDatum


\rule[-0.2cm]{5.5cm}{0.5pt} %Line

\textsc{\autorFullNameOne}
\clearpage


%Anhang ---------------------------------------------------------------------
\clearpage
\appendix
\pagenumbering{roman}
\section{Anhang}
\subsection{Kapitel im Anhang}

\subsubsection{Unterkapitel im Anhang}

Hier könnte Ihre Werbung stehen.

\addAbbildungQuelle{0.5}{Vereinfachtes Organigramm der Bundesagentur für Arbeit (Zentrale) im Anhang}{abb:Vereinfachte_Organigramm_BA}{02_Anhang/Anhang_Vereinfachte_Organigramm_BA.png}{In Anlehnung an \cite{02Kategorie_QUELLE_OrganigrammderZentrale}}

\addTabelleScaleTextQuelle{normalsize}{Vereinfachte Zeitplanung im Anhang}{tab:VereinfachteZeitplanung}{02_Anhang/Anhang_VereinfachteZeitplanung.tex}{Eigene Darstellung}



\end{document}

