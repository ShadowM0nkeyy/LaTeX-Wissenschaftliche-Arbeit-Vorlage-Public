%Seitenränder
\setlength{\topskip}{\ht\strutbox} % behebt Warnung von geometry
\geometry{a4paper,left=25mm,right=20mm,top=25mm,bottom=35mm}

\usepackage[
	automark, % Kapitelangaben in Kopfzeile automatisch erstellen
	headsepline, % Trennlinie unter Kopfzeile
	ilines % Trennlinie linksbündig ausrichten
]{scrlayer-scrpage}


% Kopfzeile 	---------------------------------------------------------------------
\pagestyle{scrheadings}
\clearpairofpagestyles

%Linker Teil der Kopfzeile
\renewcommand{\headfont}{\normalfont} % Schrift der Kopfzeile
\ihead{\large{\projektKurztitel}\\
	\begin{minipage}{0.93\textwidth}
		\small{\projektLangtitel} \\
	\end{minipage} \\
	\headmark}

%Rechter Teil der Kopfzeile
\chead{}
\ohead{\includegraphics[scale=0.12]{\betriebHSHarzShort}}
\setlength{\headheight}{16.2mm} % Höhe der Kopfzeile

% Fußzeile----------------------------------------------------------------------
\ifoot{\autorFullNameOne}
\cfoot{}
\ofoot{\pagemark}

% Schusterjungen und Hurenkinder vermeiden
\clubpenalty = 10000
\widowpenalty = 10000
\displaywidowpenalty = 10000

% Vermeidet das Fußnoten aufgeteilt werden!
\interfootnotelinepenalty=10000


% Abstand zwischen Nummerierung und Überschrift definieren
\newcommand{\colTableLightBlueSpace}{1.5cm}

% Abschnittsüberschriften im selben Stil wie beim Inhaltsverzeichnis einrücken
\renewcommand*{\othersectionlevelsformat}[3]{
	\makebox[\colTableLightBlueSpace][l]{#3\autodot}
}

%Einrückung des Inhaltsverzeichnisses
\RedeclareSectionCommand[tocindent=0em,tocnumwidth=1em]{section}
\RedeclareSectionCommand[tocindent=1em,tocnumwidth=2em]{subsection}
\RedeclareSectionCommand[tocindent=3em,tocnumwidth=3em]{subsubsection}


%Aufzählungszeichen definieren
\renewcommand{\labelitemi}{\small$\blacktriangleright$}
\renewcommand{\labelitemii}{\small$\blacktriangleright$}