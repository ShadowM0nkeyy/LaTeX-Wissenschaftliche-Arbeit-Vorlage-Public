%Für Sonderzeichen und Umlaute
\usepackage[T1]{fontenc}
\usepackage[utf8]{inputenc}

%Schrift ändern
\usepackage[scaled]{uarial} %Klon von Arial - mit Serifen

\renewcommand\familydefault{\sfdefault} %Default-Schrift auf "ohne Serifen" umstellen

% Schriftgröße zulassen
\usepackage{anyfontsize}

% Dokument auf Deutsch umstellen
\usepackage{babel}

% Tabellen
\PassOptionsToPackage{table}{xcolor}
\usepackage{tabularx}
\usepackage{colortbl}
\usepackage{float} %Damit kann man den Parameter "H" bei der table Umgebung verwenden und die Tabelle bleibt an der Stelle.

% Tabellen Linien formatieren
\usepackage{tabularx}
\usepackage{bigstrut}

% Einfache Definition der Zeilenabstände und Seitenränder etc.
\usepackage{setspace}
\usepackage{geometry}
%\usepackage[showframe]{geometry} %Zeigt einen Rahmen um den Text. Für Überlappungen.

% Einstellungen für PDF-Dateien
\usepackage{pdfpages}
\usepackage[
    bookmarks,
    bookmarksnumbered,
    bookmarksopen=true,
    bookmarksopenlevel=1,
    colorlinks=true,
    % di%ese Farbdefinitionen zeichnen Links im PDF farblich aus
    anchorcolor=MainColor,% Ankertext
    citecolor=MainColor, % Verweise auf Literaturverzeichniseinträge im Text
    filecolor=MainColor, % Verknüpfungen, die lokale Dateien öffnen
    menucolor=MainColor, % Acrobat-Menüpunkte
    urlcolor=MainColor,
    linkcolor=MainColor,
    %%
    pdftex,
    plainpages=false, % zur korrekten Erstellung der Bookmarks
    pdfpagelabels=true, % zur korrekten Erstellung der Bookmarks
    hypertexnames=false, % zur korrekten Erstellung der Bookmarks
    linktoc=page,
]{hyperref}
\hypersetup{
    pdftitle={\projektKurztitel -- \projektLangtitel},
    pdfauthor={\autorFullNameOne},
    pdfcreator={\autorFullNameOne},
    pdfsubject={\projektKurztitel -- \projektLangtitel},
    pdfkeywords={\projektKurztitel -- \projektLangtitel},
}

% Einbinden von Bilder (.jpg || .png) ermöglichen
\usepackage{graphicx}

\usepackage[automake, acronym, toc, nonumberlist, nopostdot, nogroupskip, style=super]{glossaries}
\makeglossaries

%Fußzeile und Kopfzeile 
\usepackage{chngcntr} % fortlaufendes Durchnummerieren der Fußnoten

%Literaturverzeichnis
\usepackage[babel,german=quotes]{csquotes} %Empfohlen
\usepackage[backend=biber,
    style=authoryear,isbn=false
]{biblatex}

%Löscht den Inhalt des Feldes 'note'
\AtEveryBibitem{%
    \clearfield{note}%
}

%URL-Umbrüche anpassen
\setcounter{biburllcpenalty}{7000}
\setcounter{biburlucpenalty}{8000}

\addbibresource{04_Literaturverzeichnis/Literaturverzeichnis.bib}

% Für die Darstellung eines Euro-Zeichens.
\usepackage{eurosym}

% Für das Titelblatt 
\usepackage[pages=some]{background}

%Einbinden von einer PDF-Datei
\usepackage{pdfpages}

%Sonstiges
\usepackage{xspace} % sorgt dafür, dass Leerzeichen hinter parameterlosen Makros nicht als Makroendezeichen interpretiert werden

\usepackage[normalem]{ulem} % unterstreichen [\uline], durchstreichen [\sout] etc.